\section{Il Progetto IEEE 802}

Il progetto \textbf{IEEE 802} ha l’obiettivo di definire gli \textbf{standard per le reti locali (LAN)}, sia cablate che wireless, ed è stato successivamente esteso a:

\begin{itemize}
    \item \textbf{MAN} (Metropolitan Area Network), es. \texttt{IEEE 802.16} (WiMAX)
    \item \textbf{WPAN} (Wireless Personal Area Network), es. \texttt{IEEE 802.15} (Bluetooth)
\end{itemize}

\subsection*{Struttura della pila ISO/OSI nei livelli inferiori}
IEEE 802 si occupa dei \textbf{primi due livelli} del modello ISO/OSI:
\begin{enumerate}
    \item \textbf{Livello fisico (Physical Layer)}
    \item \textbf{Livello di collegamento dati (Data Link Layer)}
\end{enumerate}

Il livello Data Link viene ulteriormente suddiviso in:
\begin{itemize}
    \item \textbf{LLC (Logical Link Control)} -- definito dallo standard \texttt{IEEE 802.2}, è comune a tutte le tecnologie e fornisce un'interfaccia standard ai livelli superiori.
    \item \textbf{MAC (Media Access Control)} -- specifico per ciascuna tecnologia, definisce come accedere al mezzo trasmissivo.
\end{itemize}

\begin{figure}[h!]
    \centering
    \includegraphics[width=1\textwidth]{images/ieee802.png}
    \caption{Struttura della pila ISO/OSI secondo IEEE 802}
    \label{fig:ieee802}
\end{figure}

\subsection*{IEEE 802.1}
Lo standard \texttt{IEEE 802.1} definisce le \textbf{caratteristiche generali del progetto}, comprese le specifiche per la struttura degli \textbf{indirizzi MAC}.



\subsection*{Vantaggi della suddivisione LLC/MAC}
\begin{itemize}
    \item Favorisce la \textbf{modularità} e l'interoperabilità tra diversi tipi di rete.
    \item I protocolli di livello superiore (es. IP, TCP, UDP) possono utilizzare una singola interfaccia LLC, indipendentemente dallo standard MAC sottostante.
    \item Facilita l'\textbf{evoluzione tecnologica}, permettendo l'introduzione di nuovi standard MAC senza modificare gli strati superiori.
\end{itemize}

\subsection{Livello LLC} Il livello LLC fornisce un'interfaccia unificata tra il livello Data Link e il livello Network (es. IP), indipendentemente dallo standard MAC sottostante (Ethernet, Wi-Fi, etc.).

\subsubsection{Frame LLC}
spiegazione

